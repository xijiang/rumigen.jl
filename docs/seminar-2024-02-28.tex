\documentclass{beamer}
\usepackage{hyperref}
\usepackage{amsmath,amssymb,amsfonts}
\usepackage{graphicx}
\usepackage{mathtools}
\usepackage{bookmark}
\usepackage{wasysym}
\usefonttheme{professionalfonts,serif}
\renewcommand{\baselinestretch}{1.25}
\usetheme{Warsaw}

\AtBeginSection[]
               {
                 \begin{frame}
                   \frametitle{Contents}  %\insertsection to display section title                                                                             
                   \tableofcontents[currentsection,hideothersubsections]
                 \end{frame}
               }
\setbeamercovered{invisible}

\begin{document}

\title[Using A, G, I for genomic prediction \& management]{Using Alternative Relationship Matrices for Genomic Prediction and Managing Genetic Diversity}
\author[Xijiang Yu, PB \& TM]{Xijiang Yu, Peer Berg and Theo Meuwissen}
\date{Feb. 28, 2024}
\institute{IHA, NMBU, \AA s, Norway}
\titlegraphic{\includegraphics[height=.09\textwidth]{figures/nmbu.jpg}\phantom{gap}\includegraphics[height=.09\textwidth]{figures/rumigen.png}}

\frame{
  \titlepage
}

\section{Background}
\subsection{A dilemma of modern breeding}
\begin{frame}{A dilemma of modern breeding}
  \begin{itemize}
  \item Needs faster genetic gain to feed increasing human demands.
  \item Must maintain genetic diversity to cope with future challenges, e.g.,
    \begin{itemize}
    \item Climate change
    \item Emerging diseases
    \item Societal demands
    \end{itemize}
  \end{itemize}
\end{frame}
% intro rumigen 

\subsection{Management of population genetic structure}
\begin{frame}{Management of population genetic structure}
  \begin{itemize}
  \item Constrain inbreeding with a relationship matrix while selecting for maximal genetic gain
    \begin{itemize}
    \item Meuwissen, 1997 %378 citations and counting
    \end{itemize}
  \item Before the genomic era:
    \begin{itemize}
    \item Pedigree-based relationship matrix $\to$ NRM (Numerator RM)
    \end{itemize}
  \item In the genomic era:
    \begin{itemize}
    \item Genomic relationship matrix $\to$ e.g., GRM, IRM (IBD RM)
    \end{itemize}
  \end{itemize}
\end{frame}

\subsection{Objectives}
\begin{frame}{Objectives}
  \begin{itemize}
  \item Utilize GRM, IRM (new tools) and NRM (old tool) to investigate
    \begin{itemize}
    \item Genetic progress
    \item Diversity, inbreeding, and future selection potentials
    \end{itemize}
  \end{itemize}
\end{frame}

\section{Materials and Methods}

\subsection{Simulation}
\begin{frame}{Simulation}
  Common base population:

  $$\boxed{\text{MaCS}}\Rightarrow\boxed{\text{25\mars}\times\text{50\female}}\xrightleftharpoons[\text{Random select}]{\text{5 generations}}\boxed{\text{200 offspring}}$$

  non-OCS:

  $$\overset{\text{Select on EBV}}\Longrightarrow\boxed{\text{25\mars}\times\text{50\female}}\xrightleftharpoons[\text{Select on EBV}]{\text{20 generations}}\boxed{\text{200 offspring}}$$

  or OCS,

  $$\overset{\text{Select on EBV}}\Longrightarrow\boxed{\text{\mars s}\times\text{\female s}}\xrightleftharpoons[\text{Select on EBV}]{\text{20 generations}}\boxed{\text{200 offspring}}$$
\end{frame}

\begin{frame}{Simulation -cont.}
  \begin{itemize}
  \item Data to record:
    \begin{itemize}
    \item Genotypes
      \begin{itemize}
      \item $50k$ chip (visible) SNP
      \item $10k$ reference (invisible) SNP
      \item $10k$ QTL
      \item uniquely coded to trace every allele
      \end{itemize}
    \item Pedigree, including phenotypes, TBV, inbreeding.
    \end{itemize}
  \end{itemize}
\end{frame}

\begin{frame}{Simulation -cont.}
  \begin{itemize}
  \item Software
    \begin{itemize}
    \item Written in Julia
    \item Started since the GEAS project
    \end{itemize}
  \item Features
    \begin{itemize}
    \item Fast, flexible, reusable and of large capacity
    \item Can deal with complex breeding structure
    \item Many structured parameter recordings and analysis tools
    \end{itemize}
  \item Availability
    \begin{itemize}
    \item \href{https://github.com/xijiang/rumigen.jl}{https://github.com/xijiang/rumigen.jl}
    \item Evolves all the time
    \item Can be tailored for other uses.
    \end{itemize}
  \end{itemize}
\end{frame}

\subsubsection{Breeding ceiling}
\begin{frame}{Breeding ceiling}
  $$\mathbf{a}\sim\text{MVN}(\mathbf{0},\mathbf{I}\sigma_{\text{tiny}}^2)\Longrightarrow\text{Ceiling} = \frac{\sum 2a_i}{\sigma_{\text{TBV}_{t=0}}}\quad\forall a_i>0$$

  \begin{itemize}
  \item A ceiling lowers when a QTL is fixed on the allele of unfavorite effect
  \item Can be obtained empirically, e.g.,
    \begin{itemize}
    \item $\sim$\phantom{1}46 $\sigma_{\text{TBV}}$ of generation $t=0$ for \phantom{0}1 $k$ QTL
    \item $\sim$146 $\sigma_{\text{TBV}}$ of generation $t=0$ for 10 $k$ QTL
    \end{itemize}
  \end{itemize}
\end{frame}

\subsection{Models}
\begin{frame}{Models}
  \begin{block}{Phenotypes}
    \begin{itemize}
    \item $\text{TBV}_i=\mathbf{a'g}_i$
      \begin{itemize}
      \item Normalized to mean 0 and variance 1.
      \end{itemize}
    \item $\text{phenotypes} = \text{TBV} + \mathbf{e}$
      \begin{itemize}
      \item Male phenotypes masked
      \end{itemize}
    \item where
      \begin{itemize}
      \item $\mathbf{a}\sim\text{MVN}(\mathbf{0},\,\mathbf{I}\sigma_a^2)$
      \item $\mathbf{e}\sim\text{MVN}(\mathbf{0},\,\mathbf{I}\sigma_a^2\frac{1-h^2}{h^2})$
      \end{itemize}
    \end{itemize}
  \end{block}
  \begin{block}{BV estimation}
    \begin{itemize}
    \item $y_{kj} = \mu+t_k+\text{EBV}_{kj}+e_{kj}$
    \end{itemize}
  \end{block}
\end{frame}

\subsection{Selection schemes}
\begin{frame}{Selection schemes}
  \begin{block}{Optimum contribution selection (OCS)}
    $$\begin{matrix}
      \text{Maximize:} & \mathbf{c}'\text{EBV}\\
      \text{Subject to:} & \mathbf{c'Xc}/2\le F_{t+1}\\
      & \mathbf{c's} = 0.5\\
      & \mathbf{c'd} = 0.5\\
      & c_i \ge 0
    \end{matrix}$$
  \end{block}
  \begin{block}{Selection schemes}
    \begin{itemize}
    \item Non-OCS
      \begin{itemize}
      \item A-, G-, I-BLUP and random selection
      \end{itemize}
    \item OCS
      \begin{itemize}
      \item AA-, AG-, GG-, IG-, and II-BLUP
      \end{itemize} 
    \end{itemize}
  \end{block}
\end{frame}

\section{Results}

\subsection{Genetic improvement}
\begin{frame}{Selection ceiling and genetic improvement}
  \includegraphics[width=\textwidth]{figures/figure-1.pdf}
\end{frame}
% Notes:
% 1. GGBLUP has the highest genetic gain, but also the lowest ceiling
% 2. Pay attention to I constrained OCS schemes. AA is also ok, but no return.
% 3. OCS schemes always have higher delta G than non-OCS ones

\begin{frame}{Number of parents selected in OCS schemes}
  \includegraphics[width=\textwidth]{figures/figure-6.pdf}
\end{frame}
% Notes:
% 1. solid line: number of males selected; dashed line: females
% 2. G as constraint selects much fewer solutions. Partly explains higher delta G and lower ceiling
% 3. G has more variations, such that it think fewer solutions are enough

\subsection{Genetic diversity}
\begin{frame}{Accumulated QTL allele losses}
  \includegraphics[width=\textwidth]{figures/figure-3.pdf}
\end{frame}
% Notes:
% 1. 4 lines in the random scheme
% 2. AGBLUP has many loci fixed and is as bad as ABLUP, which can be explained later.
% - codes in NMBU one drive rumigen 2024 jan. 15
% remove low mafs
% need redo the random scheme !!!

\begin{frame}{Genetic and genic variances}
  \includegraphics[width=\textwidth]{figures/figure-4.pdf}

  \tiny{Genetic var. $=\sigma_{\text{TBV}}^2$, genic var. $=\sum_i^{N_{\text{QTL}}}2p_iq_ia_i^2$}
\end{frame}
% Notes:
% 1. Variance reduces in line with the ceiling
% 2. Ceiling is maybe a better measure for potential

\begin{frame}{Inbreeding increment}
  \includegraphics[width=\textwidth]{figures/figure-5.pdf}
\end{frame}
% Notes:
% 1. II, and IG reached the target inbreeding level.
% 2. Pedigree inbreeding calculation has a system bias
% 3. A inbreeding is not picking up the difference between the sibs
%    - e.g., for I-, G-BLUP, the IBD inbreeding is higher.

\begin{frame}{QTL allele frequencies}
  \includegraphics[width=\textwidth]{figures/figure-7.pdf}
\end{frame}
% Notes:
% 1. Fixed loci numbers are excluded
% 2. The gap between t20 and t0 are the number of QTL fixed
% 3. AGBLUP has a bad shape, which is prone to future fixation
% - change line colors and x-, y-axis labels. <--- done
% - add random scheme.
% - figure from 2023-10-23

\subsection{$\Delta$TBV costs}
\begin{frame}{$\Delta$TBV costs}
  \begin{center}
    \begin{tabular}{rrrccc}
      \hline
      %Scheme & $\Delta\overline{\text{TBV}}$ & $\oveline{\text{Ceiling}}$ & $\Delta\overline{\text{TBV}}/\Delta\text{Ceiling}$(\%) & $\Delta\overline{\text{TBV}}/\Delta\bar{F}$ & $\Delta\overline{\text{Ceiling}}/\Delta\overline{\text{TBV}}$\\
      Scheme & $\Delta\overline{\text{TBV}}$ & $\overline{\text{Ceiling}}$ & $\frac{\Delta\overline{\text{TBV}}}{\Delta\overline{\text{Ceiling}}}$\% & $\frac{\Delta\overline{\text{TBV}}}{\Delta F}$ & $\frac{\Delta\overline{\text{Ceiling}}}{\Delta\overline{\text{TBV}}}$\\
      \hline
       A - &  7.1 &  91.9 & 16.4 & 20.1 & 6.1\\
       I - &  8.1 &  96.6 & 21.0 & 21.0 & 4.8\\
      GG - & 14.0 &  72.9 & 22.5 & 27.5 & 4.4\\
       G - &  8.4 &  97.9 & 22.6 & 25.6 & 4.4\\
      AG - & 12.4 &  96.5 & 32.2 & 35.6 & 3.1\\
      AA - & 10.5 & 104.7 & 34.6 & 42.5 & 2.9\\
      IG - & 11.4 & 105.6 & 38.9 & 52.3 & 2.6\\
      II - & 11.0 & 107.0 & 39.1 & 50.0 & 2.6\\
      \hline
    \end{tabular}
  \end{center}
\end{frame}

% \begin{frame}{Empirical probability of loci fixation against MAF and inbreeding}
%   \includegraphics[width=\textwidth]{figures/pfix.pdf}
% \end{frame}
% % function empfix in rumigen.jl -> util.jl, deprecated. a binomial sampling process.

\section{Conclusions}
\begin{frame}{Conclusions}
  \begin{itemize}
  \item The OCS schemes have higher genetic gain than the non-OCS ones
    \begin{itemize}
      \item IBD relationships constrained schemes are more ideal for their genetic gain.
    \end{itemize}
  \item I- constrained schemes have maintained more diversity.
    \begin{itemize}
    \item Higher ceiling
    \item More genetic, or genic variance
    \item Less inbreeding increase
    \item Less allele fixation
    \item Flatter QTL allele frequency distributions
    \end{itemize}
  \end{itemize}
\end{frame}
% start with the delta g per ceiling or inbreeding

\begin{frame}{Conclusions -cont.}
  \begin{itemize}
    \item Selection without constraints (traditional methods) are not only slow,
      \begin{itemize}
        \item but also costly.
      \end{itemize}
    \item Results of AA-scheme are also ok.
      \begin{itemize}
        \item Lack the advantages of genomic selection
      \end{itemize}
    \item GG-scheme found the fewest solutions.
      \begin{itemize}
        \item May due of more variable.
      \end{itemize}
    \end{itemize}
\end{frame}

\begin{frame}{Acknowledgements}
  \begin{itemize}
    \item We thank the RUMIGEN project supported by ERC
    \item XY thanks Dr. Lu Cao for many discussions of these slides
    \item Inputs from the audience
  \end{itemize}
  
  \vspace{1cm}
  
  \begin{center}
    \includegraphics[height=.15\textwidth]{figures/nmbu.jpg}\phantom{gap}\includegraphics[height=.15\textwidth]{figures/rumigen.png}  
  \end{center}
\end{frame}

\end{document}
